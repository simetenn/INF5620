\documentclass[12pt, norsk, a4paper]{article}
\usepackage{mathptmx}
\usepackage[utf8] {inputenc}
\usepackage[T1] {fontenc}
\usepackage{babel, mathpple, textcomp, varioref}
\usepackage{amssymb, amsmath, amsfonts}
\usepackage{txfonts}
\usepackage{graphicx}
\usepackage{listings}
\usepackage{pdflscape}

\tolerance = 5000
\hbadness = \tolerance
\pretolerance = 2000

 
\title{INF5620 - Mandatory Exercise 1 - Skydiving}
\author{Ingrid Elgsaas-Vada \& Simen Tennøe \\
        ingrielg \& simenten \\
        \texttt{ingrielg@student.matnat.uio.no} \texttt{simetenn@gmail.com}}
        
\begin{document}
\maketitle{}
\section*{(a)}
\begin{align*}
F_{tot} & = ma \\
F_{tot} &= F_g + F_d^{q} + F_b \\
\intertext{We add an extra term $F_s$ to be used for manipulating the problem
    to suit a spesific solution. }
ma &= -mg -\frac{1}{2} C_D \rho A |v|v + \rho g V + F_s \\
m \frac{dv}{dt} &= -mg - \frac{1}{2}C_D \rho A |v|v + \rho g V + F_s \\
\frac{dv}{dt} &= -g -\frac{C_D \rho A}{2m} + \frac{\rho g V}{m} + \frac{1}{m}F_s
\\
\intertext{To make the expression simpler we define the constants}
a &= \frac{C_D \rho A}{2m} \\
b &= g \left(\frac{\rho V}{m} -1 \right) \\
c &= \frac{1}{m}
\intertext{Using these constants we get the expression}
v'(t) &= -a |v|v + b + c F_s 
\intertext{for the ODE}
\end{align*}
\section*{(b)}
We now need to calculate the numerical scheme for this ODE. We use the
Crank-Nicolson scheme. And $a$, $b$ and $c$ are constants.
\begin{align*}
\frac{v^{n+1}-v^n}{\Delta t} &= -a (|v|v)^{n+\frac{1}{2}} + b +
cF_s^{n+\frac{1}{2}} \\
\intertext{We use a geometric average to calculate $(|v|v)^{n+\frac{1}{2}}$}
v^{n+1}-v^n &= -a \Delta t |v^n|v^{n+1} + b \Delta t + c \Delta t F_s^{n+\frac{1}{2}} \\
v^{n+1} +a \Delta t |v^n|v^{n+1} &= v^n +b \Delta t + c \Delta t
F_s^{n+\frac{1}{2}} \\
v^{n+1}(1+a\Delta t |v^n|) &= v^n +b \Delta t + c \Delta t F_s^{n+\frac{1}{2}} \\
v^{n+1} &= \frac {v^n +b \Delta t + c \Delta t F_s^{n+\frac{1}{2}}}{1+a\Delta t
    |v^n|} \\
\end{align*}
\section*{(c)}
We now assume that the ODE has a solution $\alpha t + \beta$ and we need to
prove that this does not fit the discrete equation. \\
We can show this using a subset of linear equations where $\beta=0$.
We use the discrete equation 
\[\frac{v^{n+1}-v^n}{\Delta t} = -a (|v|v)^{n+\frac{1}{2}} + b \]
If a linear equation $\alpha t$ is the solution to this equation then both
sides will be constants. The left hand side is the discrete definition of the
first order derivative and as such is obviously constant and equal to $\alpha$.
\[\frac{v^{n+1}-v^n}{\Delta t} = \frac{\alpha t^{n+1}-\alpha t^n}{\Delta t} =
\frac{\alpha((n+1)\Delta t - n \Delta t)}{\Delta t} = \frac{\alpha \Delta t
    (n+1-n)}{\Delta t} = \alpha \]
Now we need to see that if the same holds for the right hand side of the
equation.
\begin{align*}
-a |v^n|v^{n+1} + b &= -a |\alpha t^n|\alpha t^{n+1} + b \\
        &= -a |\alpha n \Delta t|\alpha (n+1) \Delta t + b \\
\intertext{We know that both $n$ and $\Delta t$ are positive}
&= -a n(n+1) \Delta t^2 \alpha |\alpha| + b \\
\intertext{So the left hand side of the equation changes with n and is
    therefore not constant so a linear equation is not a solution to the
        discrete equation.}
\end{align*}
To make a linear equation the solution to our problem we need to manipulate
$F_s$. We still use the linear equation $\alpha t$ as $\beta$ is the initial
condition and in our problem this is 0. 
\begin{align*}
-a |v^n|v^{n+1} + b + cF_s^{n+\frac{1}{2}} & = \alpha \\
-a |\alpha t^n|\alpha t^{n+1} + b + cF_s^{n+\frac{1}{2}} &= \alpha \\
cF_s^{n+\frac{1}{2}} &= \alpha + a |\alpha n \Delta t|\alpha (n+1) \Delta t - b
\\
cF_s^{n+\frac{1}{2}} &= \alpha + a n(n+1)\Delta t^2 |\alpha| \alpha -b \\
F_s^{n+\frac{1}{2}} &= \frac{\alpha + a n(n+1)\Delta t^2 |\alpha| \alpha -b }{c}
\end{align*}
\end{document}
